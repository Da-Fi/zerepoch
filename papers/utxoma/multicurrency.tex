\section{Multi-Asset Support}
\label{sec:multicurrency}

In Bitcoin's ledger model~\cite{Nakamoto,formal-model-of-bitcoin-transactions,Zahnentferner18-UTxO}, transactions spend as yet \emph{unspent transaction outputs \textup{(}\!\UTXO{}s\textup{)}}, while supplying new unspent outputs to be consumed by subsequent transactions.
Each individual \UTXO\ locks a specific \emph{quantity} of cryptocurrency by imposing specific conditions that need to be met to spend that quantity, such as for example signing the spending transaction with a specific secret cryptographic key, or passing some more sophisticated conditions enforced by a \emph{validator script}.
Quantities of cryptocurrency in a transaction output are represented as an integral number of the smallest unit of that particular cryptocurrency --- in Bitcoin, these are Satoshis.
To natively support multiple currencies in transaction outputs, we generalise those integral quantities to natively support the dynamic creation of new user-defined \emph{assets} or \emph{tokens}. Moreover, we require a means to forge tokens in a manner controlled by an asset's \emph{forging policy}.

We achieve all this by the following three extensions to the basic \UTXO\ ledger model that are further detailed in
the remainder of this section.
%
\begin{enumerate}
\item Transaction outputs lock a \emph{heterogeneous token bundle} instead of only an integral value of one cryptocurrency.
\item We extend transactions with a \emph{forge} field. This is a token bundle of tokens that are created (minted) or destroyed (burned) by that transaction.
\item We introduce \emph{forging policy scripts \textup{(}FPS\textup{)}} that govern the creation and destruction of assets in forge fields. These scripts are not unlike the validators locking outputs in \UTXO.
\end{enumerate}

\subsection{Token bundles}

We can regard transaction outputs in an \UTXO\ ledger as pairs \((\val, \nu)\) consisting of a locked value $\val$ and a validator script $\nu$ that encodes the spending condition. The latter may be proof of ownership by way of signing the spending transaction with a specific secret cryptography key or a temporal condition that allows an output to be spent only when the blockchain has reached a certain height (i.e. a certain number of blocks have been produced).

To conveniently use multiple currencies in transaction outputs, we want each output to be able to lock varying quantities of multiple different currencies at once in its $\val$ field.
This suggests using finite maps from some kind of \emph{asset identifier} to an integral quantity as a concrete representation, e.g. $\texttt{Coin} \mapsto 21$.
Looking at the standard \UTXO\ ledger rules~\cite{Zahnentferner18-UTxO}, it becomes apparent that cryptocurrency quantities need to be monoids.
It is a little tricky to make finite maps into a monoid, but the solution is to think of them as \emph{finitely supported functions} (see Section~\ref{sec:fsfs} for details).

If want to use \emph{finitely supported functions} to achieve a uniform
representation that can handle groups of related, but \emph{non-fungible} tokens, we need to go a step further.
In order to not lose the grouping of related non-fungible tokens (all house tokens issued by a specific entity, for example) though, we need to move to a two-level structure --- i.e., finitely-supported functions of finitely-supported functions. Let's consider an example. Trading of rare in-game items is popular in modern, multi-player computer games. How about representing ownership of such items and trading of that ownership on our multi-asset \UTXO\ ledger? We might need tokens for ``hats'' and ``swords'', which form two non-fungible assets with possibly multiple tokens of each asset --- a hat is interchangeable with any other hat, but not with a sword, and also not with the currency used to purchase these
items. Here our two-level structure pays off in its full generality, and we can represent currency to purchase items together with sets of items, where some can be multiples, e.g.,
%
\begin{align*}
  & \{\mathsf{Coin} \mapsto \{\mathsf{Coin} \mapsto 2\}, \mathsf{Game} \mapsto \{\mathsf{Hat} \mapsto 1, \mathsf{Sword} \mapsto 4\}\} \\
  + \ & \{\mathsf{Coin} \mapsto \{\mathsf{Coin} \mapsto 1\}, \mathsf{Game} \mapsto \{\mathsf{Sword} \mapsto 1, \mathsf{Owl} \mapsto 1\}\} \\
  = \ & \{\mathsf{Coin} \mapsto \{\mathsf{Coin} \mapsto 3\}, \mathsf{Game} \mapsto \{\mathsf{Hat} \mapsto 1, \mathsf{Sword} \mapsto 5, \mathsf{Owl} \mapsto 1\}\} \ .
\end{align*}

\subsection{Forge fields}

If new tokens are frequently generated (such as issuing new hats whenever an in-game achievement has been reached) and destroyed (a player may lose a hat forever if the wind picks up), these operations need to be lightweight and cheap. We achieve this by adding a forge field to every transaction. It is a token bundle (just like the $\val$ in an output), but admits positive quantities (for minting new tokens) and negative quantities (for burning existing tokens). Of course, minting and burning needs to be strictly controlled.

\subsection{Forging policy scripts}

The script validation mechanism for locking \UTXO\ outputs is as follows :
in order to for a transaction to spend an output \((\val, \nu)\), the validator
script $\nu$ needs to be executed and approve of the spending transaction.
Similarly, the forging
policy scripts associated with the tokens being minted or burned by a transaction
are run in order to validate those actions.
In the spirit of the Bitcoin Miniscript approach, we chose to include a simple
scripting language supporting forging policies for several common usecases, such as
single issuer, non-fungible, or one-time issue tokens, etc.
(see Section~\ref{sec:fps-language} for all the usecases).

In order to establish a permanent association between the forging policy and the
assets controlled by it, we propose a hashing approach, as opposed to a global registry
lookup. Such a registry requires a specialized access control scheme, as well
as a scheme for cleaning up unused entries.
In the representation of custom assets we propose, each token is associated with the
hash of the forging policy script required to validate at the time of forging
the token, eg.
in order to forge the value
\(\{\mathsf{HASHVALUE} \mapsto \{\mathsf{Owl} \mapsto 1\}\}\), a script whose
hash is $\mathsf{HASHVALUE}$ will be run.

Relying on permanent hash associations to identify asset forging policies and their assets also has its disadvantages.
For example, policy hashes are long strings that, in our model, will have multiple copies stored on the ledger.
Such strings are not human-readable, take up valuable ledger real estate, and increase transaction-size-based fees.
