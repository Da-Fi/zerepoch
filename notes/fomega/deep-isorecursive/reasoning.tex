\documentclass{article}

\usepackage{amsmath,amssymb,amsthm,stmaryrd}
\usepackage{enumerate}
\usepackage{mathpartir}

\renewcommand{\:}{\mathbin{:}}

\title{A Type System With Iso-Recursive Types and Higher Kinds That We Can Mimic in Zerepoch Core}

\begin{document}

\maketitle

Figures 1-5 introduce the relevant part of the syntax and inference rules for the system found in \cite{Dreyer05}. We are interested in this system because it is an extension of system F$_\omega$ with iso-recursive types (among other things). Specifically, the system has a ``coercion'' base type $C_1 \rightsquigarrow C_2$ along with familiar coercion functions {\sf fold}$_C$ and {\sf unfold}$_C$ at the term level. The typing rules for these coercions can be found near the bottom of Figure 5. An interesting feature of the coercion functions is that they can coerce fixed point types inside of an elimination context $\mathcal{E}$ (see Figure 4).

We can use this style iso-recursive types in Zerepoch Core with minimal modification to the language. First, we discard the coercion type $C_1 \rightsquigarrow C_2$ in favour of the usual function type $C_1 \longrightarrow C_2$. While the special coercion type may allow for easier erasure of coercions at rutime, it would require us to expand the syntax of Zerepoch Core, which we want to avoid as much as possible. Additionally, we present {\sf fold} and {\sf unfold} as term formation operators instead of functions, to better match the existing syntax of Zerepoch Core. We also remove the dependent projections from elimination contexts, since Zerepoch Core does not have dependent pair kinds. Finally, we observe that in the simpler kind system of Zerepoch Core, $\Delta \vdash Q \uparrow \textbf{T}$ and $\Delta \vdash Q \: \textbf{T}$ correspond to the same requirement, so that $Q$ expands if and only if $\Delta \vdash Q \: \textbf{T}$ and $Q = \mathcal{E}\{\mu\alpha\: K.C\}$. We thus arrive at inference rules:

\begin{mathpar}
  \inferrule{\Gamma \vdash Q :: ({\sf type}) \\ Q = \mathcal{E}\{({\sf fix}\,\, a\,\, C)\} \\ \Gamma \vdash M \: \mathcal{E}\{[({\sf fix}\,\, a \,\, C)/a]C\}}{\Gamma \vdash ({\sf wrap}\,\, a\,\, C\,\, M) \: Q}
  \and
  \inferrule{\Gamma \vdash Q :: ({\sf type}) \\ Q = \mathcal{E}\{({\sf fix}\,\, a \,\, C)\} \\ \Gamma \vdash M \: Q}{\Gamma \vdash ({\sf unwrap}\,\, M) \: \mathcal{E}\{[({\sf fix}\,\, a \,\, C)/a]C\}}
\end{mathpar}  

Note that in Zerepoch Core, we do not have to specify the kind of the variable $a$ in $({\sf fix}\,\, a\,\, C)$, but that in the system of \cite{Dreyer05} we must specify the kind. This is because the latter system has a far more complex kind system, while in Zerepoch Core the kind may be inferred. I further suggest that we adopt the syntax $\mathcal{E} ::= \bullet \mid [ \mathcal{E}\,\, C]$ for elimination contexts in Zerepoch Core, to match the syntax for type application there.

% Fig 3.1: Syntax 
\begin{figure}
\begin{align*}
  & \text{Constructor Variables} &\alpha, \beta &\in \textit{ConVars}\\
  & \text{Kinds} &K,L  &::= 1 \mid \textbf{T} \mid \mathfrak{s} \mid \Sigma \alpha \: K_1.K_2 \mid \Pi\alpha\:K_1.K_2 \\
  & \text{Type Constructors} &C,D  &::= \alpha \mid b \mid \langle \rangle \mid \langle \alpha = C_1,C_2 \rangle \mid \pi_iC \mid \lambda \alpha \: K.C \\
  &&& \mid C_1(C_2) \mid \mu\alpha\: K .C\\
  & \text{Base Types} & b & ::= \text{{\sf unit}} \mid C_1 \times C_2 \mid C_1 \longrightarrow C_2 \mid \forall \alpha \: K.C
  \\ &&& \mid \exists \alpha \: K .C \mid C_1 \rightsquigarrow C_2 \mid \text{{\sf maybe}}(C) \\
  & \text{Static Contexts} & \Delta & ::= \emptyset \mid \Delta,\alpha \: K\\
  & \text{Value Variables} & x,y & \in \textit{ValVars} \\
  & \text{Values} & v,w & ::= x \mid \langle \rangle \mid \text{{\sf fun} } x(x_1\:C_1)\:C_2.e \mid \Lambda\alpha\:K.e\\
  &&& \mid \text{{\sf pack} } [C,v] \text{ {\sf as} } D \mid \text{{\sf fold}}_C \mid \text{{\sf unfold}}_C \mid \text{{\sf fold}}_C\langle \langle v \rangle \rangle \\
  & \text{Terms} & e,f & ::= v \mid \pi_iv \mid v_1(v_2) \mid v[C] \mid \text{{\sf let} } [\alpha,x] = \text{ {\sf unpack} } v \text{ {\sf in} } (e\:C) \\
  &&& \mid \text{{\sf let} } \alpha = C \text{ {\sf in} } e \mid \text{{\sf let} } x = e_1 \text{ {\sf in} } e_2 \mid v_1 \langle \langle v_2 \rangle \rangle \mid \text{{\sf fetch}}(v) \mid \text{{\sf rec}}(x\:C.e)\\
  & \text{Dynamic Contexts} & \Gamma & ::= \emptyset \mid \Gamma,\alpha\:K \mid \Gamma,x\:C
\end{align*}  
\caption{Syntax}
\end{figure}

% Fig 3.4: Inference Rules for Kinds and Static Contexts
\begin{figure}
Well-formed static contexts $\Delta \vdash \text{ok}$ 
\begin{mathpar}
  \inferrule{\text{}}{\emptyset \vdash \text{ok}}
  \and
  \inferrule{\Delta \vdash K \text{ kind}}{\Delta,\alpha \: K \vdash \text{ok}}
\end{mathpar}
Well-formed kinds $\Delta \vdash K \text{ kind}$ 
\begin{mathpar}
  \inferrule{\Delta \vdash \text{ok}}{\Delta \vdash 1 \text{ kind}}
  \and
  \inferrule{\Delta \vdash \text{ok}}{\Delta \vdash \textbf{T} \text{ kind}}
  \and
  \inferrule{\Delta \vdash C \: \textbf{T}}{\Delta \vdash \mathfrak{s}(C) \text{ kind}}
  \and
  \inferrule{\Delta,\alpha \: K' \vdash K'' \text{ kind}}{\Delta \vdash \Sigma\alpha\: K'.K'' \text{ kind}}
  \and
  \inferrule{\Delta,\alpha\: K' \vdash K'' \text{ kind}}{\Delta \vdash \Pi\alpha\: K'.K'' \text{ kind}}
\end{mathpar}
%Kind equivalence $\Delta \vdash K_1 \equiv K_2$
%\begin{mathpar}
%  \inferrule{\Delta \vdash \text{ok}}{\Delta \vdash 1 \equiv 1}
%  \and
%  \inferrule{\Delta \vdash \text{ok}}{\Delta \vdash \textbf{T} \equiv \textbf{T}}
%  \and
%  \inferrule{\Delta \vdash C_1 \equiv C_2 \: \textbf{T}}{\Delta \vdash \mathfrak{s}(C_1) \equiv \mathfrak{s}(C_2)}
%  \and
%  \inferrule{\Delta \vdash K_1' \equiv K_2' \\ \Delta, \alpha \: K_1' \vdash \alpha K_1'' \equiv K_2''}
%            {\Delta \vdash \Sigma\alpha\: K_1'.K_1'' \equiv \Sigma\alpha\: K_2'.K_2''}
%  \and
%  \inferrule{\Delta \vdash K_2' \equiv K_1' \\ \Delta,\alpha\: K_2' \vdash K_1'' \equiv K_2''}
%            {\Delta \vdash \Pi \alpha \: K_1'.K_1'' \equiv \Pi \alpha \: K_2'.K_2''}
%\end{mathpar}
%Kind subtyping $\Delta \vdash K_1 \leq K_2$
%\begin{mathpar}
%  \inferrule{\Delta \vdash \text{ok}}{\Delta \vdash 1 \leq 1}
%  \and
%  \inferrule{\Delta \vdash \text{ok}}{\Delta \vdash \textbf{T} \leq \textbf{T}}
%  \and
%  \inferrule{\Delta \vdash C_1 \equiv C_2 \: \textbf{T}}{\Delta \vdash \mathfrak{s}(C_1) \leq \mathfrak{s}(C_2)}
%  \and
%  \inferrule{\Delta \vdash C \: \textbf{T}}{\Delta \vdash \mathfrak{s}(C) \leq \textbf{T}}
%  \and
%  \inferrule{\Delta \vdash K_1' \leq K_2' \\ \Delta,\alpha\: K_1' \vdash K_1'' \leq K_2'' \\ \Delta \vdash \Sigma \alpha \: K_2'.K_2'' \text{ kind}}
%            {\Delta \vdash \Sigma \alpha \: K_1' . K_1'' \leq \Sigma \alpha \: K_2' . K_2''}
%  \and
%  \inferrule{\Delta \vdash K_2' \leq K_1' \\ \Delta,\alpha\: K_2' \vdash K_1'' \leq K_2'' \\ \Delta \vdash \Pi \alpha \: K_1' . K_1'' \text{ kind}}
%            {\Delta \vdash \Pi \alpha \: K_1'.K_1'' \leq \Pi \alpha\: K_2'.K_2''}
%\end{mathpar}
\caption{Inference Rules for Kinds and Static Contexts}
\end{figure}

% Fig. 3.5: Inference Rules for Type Constructors
\begin{figure}
Well-formed constructors $\Delta \vdash C \: K$  
\begin{mathpar}
  \inferrule{\Delta \vdash \text{ok} \\ \alpha \: K \in \Delta}{\Delta \vdash \alpha \: K}
  \and
  \inferrule{\Delta \vdash \text{ok}}{\Delta \vdash {\sf unit} \: \textbf{T}}
  \and
  \inferrule{\Delta \vdash C' \: \textbf{T} \\ \Delta \vdash C'' \: \textbf{T}}
            {\Delta \vdash C' \times C'' \: \textbf{T}}
  \and              
  \inferrule{\Delta \vdash C' \: \textbf{T} \\ \Delta \vdash C'' \: \textbf{T}}
            {\Delta \vdash C' \longrightarrow C'' \: \textbf{T}}
  \and
  \inferrule{\Delta,\alpha \: K \vdash C \: \textbf{T}}{\Delta \vdash \forall \alpha \: K . C  \:\textbf{T}}
  \and
  \inferrule{\Delta,\alpha\: K \vdash C \: \textbf{T}}{\Delta \vdash \exists\alpha \: K . C \: \textbf{T}}
  \and
  \inferrule{\Delta \vdash \text{ok}}{\Delta \vdash \langle \rangle \: 1}
  \and
  \inferrule{\Delta \vdash C' \: K' \\ \Delta,\alpha \: K' \vdash C'' \: K''}
            {\Delta \vdash \langle \alpha = C' , C''\rangle \: \Sigma \alpha \: K'.K''}
  \and
  \inferrule{\Delta \vdash C \: \Sigma \alpha \: K' . K''}{\Delta \vdash \pi_1 C \: K'}
  \and
  \inferrule{\Delta \vdash C \: \Sigma \alpha \: K' . K''}{\Delta \vdash \pi_2 C \: K''[\pi_1C/\alpha]}
  \and
  \inferrule{\Delta,\alpha\: K' \vdash C \: K''}{\Delta \vdash \lambda \alpha \: K' . C \: \Pi \alpha \: K'.K''}
  \and
  \inferrule{\Delta \vdash C \: \Pi \alpha \: K'.K'' \\ \Delta \vdash D \: K'}
            {\Delta \vdash C(D) \: K''[D/\alpha]}
  \and
  \inferrule{\Delta \vdash C \: \textbf{T}}{\Delta \vdash C \: \mathfrak{s}(C)}
  \and
  \inferrule{\Delta \vdash \pi_1C \: K' \\ \Delta \vdash \pi_2C \: K''}{\Delta \vdash C \: K' \times K''}
  \and
  \inferrule{\Delta,\alpha : K' \vdash C(\alpha) \: K'' \\ \Delta \vdash C \: \Pi \alpha \: K' . L}
            {\Delta \vdash C : \Pi \alpha \: K'. K''}
  \and
  \inferrule{\Delta \vdash C \: K' \\ \Delta \vdash K' \leq K}{\Delta \vdash C \: K}
  \and
  \inferrule{\Delta,\alpha\: K \vdash C \: K}{\Delta \vdash \mu \alpha \: K.C \: K}
  \and
  \inferrule{\Delta \vdash C' \: \textbf{T} \\ \Delta \vdash C'' \: \textbf{T}}{\Delta \vdash C' \rightsquigarrow C'' \: \textbf{T}}
  \and
  \inferrule{\Delta \vdash C \: \textbf{T}}{\Delta \vdash {\sf maybe}(C) \: \textbf{T}}
\end{mathpar}
%Constructor equivalence $\Delta \vdash C_1 \equiv C_2 \: K$
%\begin{mathpar}      
%  \inferrule{\Delta \vdash C \: K}{\Delta \vdash C \equiv C \: K}
%  \and
%  \inferrule{\Delta \vdash C_2 \equiv C_1 \: K}{\Delta \vdash C_1 \equiv C_2 \: K}
%  \and
%  \inferrule{\Delta \vdash C_1 \equiv C_2 \: K \\ \Delta \vdash C_2 \equiv C_3 \: K}
%            {\Delta \vdash C_1 \equiv C_3 \: K}
%  \and
%  \inferrule{\Delta \vdash C_1' \equiv C_2' \: \textbf{T} \\ \Delta \vdash C_1'' \equiv C_2'' \: \textbf{T}}
%            {\Delta \vdash C_1' \times C_1'' \equiv C_2' \times C_2'' \: \textbf{T}}
%  \and
%  \inferrule{\Delta \vdash C_1' \equiv C_2' \: \textbf{T} \\ \Delta \vdash C_1'' \equiv C_2'' \: \textbf{T}}
%            {\Delta \vdash C_1' \longrightarrow C_1'' \equiv C_2' \longrightarrow C_2'' \: \textbf{T}}
%  \and
%  \inferrule{\Delta \vdash K_1 \equiv K_2 \\ \Delta,\alpha : K_1 \vdash C_1 \equiv C_2 \: \textbf{T}}
%            {\Delta \vdash \forall \alpha \: K_1.C_1 \equiv \forall \alpha \: K_2.C_2 \: \textbf{T}}
%  \and
%  \inferrule{\Delta \vdash K_1 \equiv K_2 \\ \Delta,\alpha \: K_1 \vdash C_1 \equiv C_2 \: \textbf{T}}
%            {\Delta \vdash \exists \alpha \: K_1.C_1 \equiv \exists \alpha \: K_2.C_2 : \textbf{T}}
%  \and
%  \inferrule{\Delta \vdash C_1' \equiv C_2' \: K' \\ \Delta,\alpha \: K' \vdash C_1'' \equiv C_2'' \: K''}
%            {\Delta \vdash \langle \alpha = C_1',C_1''\rangle \equiv \langle \alpha = C_2',C_2''\rangle \: \Sigma\alpha \: K'.K''}
%  \and
%  \inferrule{\Delta \vdash C_1 \equiv C_2 \: \Sigma \alpha \: K'.K''}{\Delta \vdash \pi_1C_1 \equiv \pi_1C_2 \: K'}
%  \and
%  \inferrule{\Delta \vdash C_1 \equiv C_2 \: \Sigma \alpha \: K'.K''}{\Delta \vdash \pi_2C_1 \equiv \pi_2C_2 \: K''[\pi_1C_1/\alpha]}
%  \and
%  \inferrule{\Delta \vdash K_1 \equiv K_2 \\ \Delta,\alpha \: K_1 \vdash C_1 \equiv C_2 \: K_1}{\Delta \vdash \mu \alpha \: K%_1 . C_1 \equiv \mu \alpha \: K_2 . C_2 \: K_1}
%  \and
%  \inferrule{\Delta \vdash C_1' \equiv C_2' \: \textbf{T} \\ \Delta \vdash C_1'' \equiv C_2'' \: \textbf{T}}{\Delta \vdash C_%1' \rightsquigarrow C_1'' \equiv C_2' \rightsquigarrow C_2'' \: \textbf{T}}
%  \and
%  \inferrule{\Delta \vdash C_1 \equiv C_2 \: \textbf{T}}{\Delta \vdash \text{{\sf maybe}}(C_1) \equiv \text{{\sf maybe}}(C_2)% \: \textbf{T}}
%\end{mathpar}
\caption{Inference Rules for Type Constructors}
\end{figure}

%...continued
%\begin{mathpar}
%  \inferrule{\Delta \vdash K_1' \equiv K_2' \\ \Delta,\alpha : K_1' \vdash C_1 \equiv C_2 \: K''}
%            {\Delta \vdash \lambda \alpha \: K_1' .C_1 \equiv \lambda \alpha \: K_2'.C_2 \: \Pi\alpha\:K_1'.K''}
%  \and            
%  \inferrule{\Delta \vdash C_1 \equiv C_2 \: \Pi \alpha \: K'.K'' \\ \Delta \vdash D_1 \equiv D_2 \: K'}
%            {\Delta \vdash C_1(D_1) \equiv C_2(D_2) \: K''[D_1/\alpha]}
%  \and
%  \inferrule{\Delta \vdash C_1 \: 1 \\ \Delta \vdash C_2 \: 1}{\Delta \vdash C_1 \equiv C_2 \: 1}
%  \and
%  \inferrule{\Delta \vdash C_1 \: \mathfrak{s}(C) \\ \Delta \vdash C_2 \: \mathfrak{s}(C)}
%            {\Delta \vdash C_1 \equiv C_2 \: \mathfrak{s}(C)}
%  \and
%  \inferrule{\Delta \vdash \pi_1C_1 \equiv \pi_1C_2 \: K' \\ \Delta \vdash \pi_2C_1 \equiv \pi_2C_2 \: K''}      
%            {\Delta \vdash C_1 \equiv C_2 \: K' \times K''}
%  \and
%  \inferrule{\Delta,\alpha \: K' \vdash C_1(\alpha) \equiv C_2(\alpha) \: K'' \\ \Delta \vdash C_1 \: \Pi \alpha \: K'.L_1 \\% \Delta \vdash C_2 \: \Pi \alpha \: K'.L_2}
%            {\Delta \vdash C_1 \equiv C_2 : \Pi\alpha \: K'.K''}
%  \and
%  \inferrule{\Delta \vdash C_1 \equiv C_2 \: K' \\ \Delta \vdash K' \leq K}
%            {\Delta \vdash C_1 \equiv C_2 \: K}
%\end{mathpar}

% Fig 3.8: Weak Head Normalization for Type Constructors
\begin{figure}
\begin{align*}
  & \text{Elimination Contexts} & \mathcal{E} & ::= \bullet \mid \mathcal{E}(C) \mid \pi_i\mathcal{E}\\
  & \text{Constructor Paths} & P & ::= b \mid \mathcal{E}\{\alpha\} \mid Q\\
  & \text{Recursive Type Paths} & Q & ::= \mathcal{E}\{\mu\alpha\: K.C\}
\end{align*}  

\begin{align*}
  & \Delta \vdash b \uparrow \textbf{T} & \\
  & \Delta \vdash \alpha \uparrow \Delta(\alpha) &\\
  & \Delta \vdash P(C) \uparrow K''[C/\alpha] & \text{if } \Delta \vdash P \uparrow \Pi\alpha \: K'.K''\\
  & \Delta \vdash \pi_1P \uparrow K' & \text{if } \Delta \vdash P \uparrow \Sigma \alpha\:K'.K''\\
  & \Delta \vdash \pi_2P \uparrow K''[\pi_1P/\alpha] & \text{if } \Delta \vdash P \uparrow \Sigma \alpha \: K' . K''
  & \Delta \vdash \mu\alpha \: K.C \uparrow K
\end{align*}  

\begin{align*}
  & \Delta \vdash \mathcal{E}\{(\lambda\alpha\:K',C)C'\} \stackrel{\text{wh}}{\longrightarrow} \mathcal{E}\{C[C'/\alpha]\} &\\
  & \Delta \vdash \mathcal{E}\{\pi_1\langle \alpha = C',C'' \rangle\} \stackrel{\text{wh}}{\longrightarrow} \mathcal{E}\{C'\} &\\
  & \Delta \vdash \mathcal{E}\{\pi_2\langle \alpha = C',C'' \rangle\} \stackrel{\text{wh}}{\longrightarrow} \mathcal{E}\{C''[C'/\alpha]\} & \\
  & \Delta \vdash P \stackrel{\text{wh}}{\longrightarrow} C & \text{if } \Delta \vdash P \uparrow \mathfrak{s}(C)
\end{align*}

\begin{align*}
  & \Delta \vdash C \stackrel{\text{wh}}{\Longrightarrow} D & \text{if } \Delta \vdash C \stackrel{\text{wh}}{\longrightarrow} C' \text{ and } \Delta \vdash C' \stackrel{\text{wh}}{\Longrightarrow} D \\
  & \Delta \vdash C \stackrel{\text{wh}}{\Longrightarrow} C & \text{otherwise}
\end{align*}
\caption{Weak Head Normalization for Type Constructors}
\end{figure}

% Fig 3.12: Inference Rules for Terms and Dynamic Contexts
\begin{figure}
Well-formed dynamic contexts $\Gamma \vdash \text{ok}$
\begin{mathpar}
  \inferrule{\text{}}{\emptyset \vdash \text{ok}}
  \and
  \inferrule{\Gamma \vdash K \text{ kind}}{\Gamma,\alpha \: K \vdash \text{ok}}
  \and
  \inferrule{\Gamma \vdash C \: \textbf{T}}{\Gamma,x\:C \vdash \text{ok}}
\end{mathpar}
Well-formed terms $\Gamma \vdash e \: C$
\begin{mathpar}
  \inferrule{\Gamma \vdash \text{ok} \\ x\:C \in \Gamma}{\Gamma \vdash x\:C}
  \and
  \inferrule{\Gamma \vdash \text{ok}}{\Gamma \vdash \langle \rangle \: {\sf unit}}
  \and
  \inferrule{\Gamma \vdash v' \: C' \\ \Gamma \vdash v'' \: C''}{\Gamma \vdash \langle v' , v'' \rangle \: C' \times C''}
  \and
  \inferrule{\Gamma \vdash C_1 \times C_2 \\ i \in \{1,2\}}{\Gamma \vdash \pi_iv \: C_i}
  \and
  \inferrule{\Gamma,x\:C' \longrightarrow C'',x'\:C' \vdash e\:C''}{\Gamma \vdash {\sf fun} x(x' \: C')\:C''.e \: C' \longrightarrow C''}
  \and
  \inferrule{\Gamma \vdash v \: C' \longrightarrow C \\ \Gamma \vdash v' \: C'}{\Gamma \vdash v(v') \: C}
  \and
  \inferrule{\Gamma,\alpha \: K \vdash e\: C}{\Gamma \vdash \Lambda\alpha\:K.e \: \forall \alpha \: K .C}
  \and
  \inferrule{\Gamma \vdash v\:\forall \alpha \: K.C \\ \Gamma \vdash D \: K}{\Gamma \vdash v[D] \: C[D/\alpha]}
  \and
  \inferrule{\Gamma \vdash D \equiv \exists\alpha\:K.C' \: \textbf{T} \\ \Gamma \vdash C \: K \\ \Gamma \vdash v \: C'[C/\alpha]}{\Gamma \vdash \text{{\sf pack} } [C,v] \text{ {\sf as} } D \: D}
  \and
  \inferrule{\Gamma \vdash v \: \exists \alpha \: K . C' \\ \Gamma, \alpha \: K,x\:C' \vdash e \: C \\ \Gamma \vdash C \: \textbf{T}}{\Gamma \vdash \text{{\sf let} } [\alpha,x] = \text{ {\sf unpack} } v \text{ {\sf in} } (e\:C) \: C}
  \and
  \inferrule{\Gamma \vdash C \: K \\ \Gamma,\alpha\: K \vdash e \: D}{\Gamma \vdash \text{{\sf let} } \alpha = C \text{ {\sf in} } e \: D[C/\alpha]}
  \and
  \inferrule{\Gamma \vdash e' \: C' \\ \Gamma,x\:C' \vdash e\:C}{\Gamma \vdash \text{{\sf let} } x = e' \text{ {\sf in} } e \: C}
  \and
  \inferrule{\Gamma \vdash e \: C' \\ \Gamma \vdash C' \equiv C \: \textbf{T}}{\Gamma \vdash e \: C}
  \and
  \inferrule{\Gamma \vdash C \equiv Q \: \textbf{T} \\ \Gamma \vdash Q \text{ expands}}{\Gamma \vdash \text{{\sf fold}}_C \: \text{expand}(Q) \rightsquigarrow Q}
  \and
  \inferrule{\Gamma \vdash C \equiv Q \: \textbf{T} \\ \Gamma \vdash Q \text{ expands}}{\Gamma \vdash \text{{\sf unfold}}_C \: Q \rightsquigarrow \text{expand}(Q)}
  \and 
  \inferrule{\Gamma \vdash v \: C' \rightsquigarrow C \\ \Gamma \vdash v' \: C'}{\Gamma \vdash v\langle \langle v' \rangle \rangle \: C'}
  \and
  \inferrule{\Gamma \vdash v \: \text{{\sf maybe}}(C)}{\Gamma \vdash \text{{\sf fetch}}(v) \: C}
  \and
  \inferrule{\Gamma,x\:\text{{\sf maybe}}(C) \vdash e \: C}{\Gamma \vdash \text{{\sf rec}}(x\:C.e) \: C}
  \and
  \inferrule{\Delta \vdash Q \: \textbf{T} \\ \Delta \vdash Q \uparrow \textbf{T} \\ Q = \mathcal{E}\{\mu\alpha\:K.C\}}{\Delta \vdash Q \text{ expands}}
\end{mathpar}
Where $\text{expand}(Q) \stackrel{\text{def}}{=} \mathcal{E}\{C[\mu\alpha \: K.C/\alpha]\}$ in case $Q = \mathcal{E}\{\mu\alpha\:K.C\}$.
\caption{Inference Rules for Terms and Dynamic Contexts}
\end{figure}

%Figure 3.14: Dynamic Semantics of the Core Language
%\begin{mathpar}
%  \inferrule{\text{}}{\pi_i\langle v_1,v_2 \rangle \mapsto v_i}
%  \and
%  \inferrule{v = \text{{\sf fun} } x(x'\:C')\:C'' \:e}{v(v') \mapsto e[v/x][v'/x']}
%  \and
%  \inferrule{\text{}}{(\Lambda\alpha \: K.e)[C] \mapsto e[C/\alpha]}
%  \and
%  \inferrule{\text{}}{\text{{\sf let} } [\alpha,x] = \text{ {\sf unpack} } (\text{{\sf pack} }[C,v] \text{ {\sf as} } D)\text%{ {\sf in} }(e\:C') \mapsto e[C/\alpha][v/x]}
%  \and
%  \inferrule{\text{}}{\text{{\sf let} } \alpha = C \text{ {\sf in} } e \mapsto e[C/\alpha]}
%  \and
%  \inferrule{e_1 \mapsto e_1'}{\text{{\sf let} } x = e_1 \text{ {\sf in} } e_2 \mapsto \text{ {\sf let} } x = e_1' \text{ {\s%f in} } e_2}
%  \and
%  \inferrule{\text{}}{\text{{\sf let} } x = v \text{ {\sf in} } e \mapsto e[v/x]}
%\end{mathpar}

\begin{thebibliography}{9}
\bibitem{Dreyer05}{D. Dreyer, Understanding and Evolving the ML Module System, PhD Thesis, Carnegie Mellon University, May 2005}
\end{thebibliography}

\end{document}


